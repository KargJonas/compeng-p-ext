% !TeX program = xelatex
% !TeX encoding = UTF-8
% !TeX spellcheck = en_US
% !BIB program = biber
%% 
%% The above lines help editors like TeXstudio to automatically choose the right tools
%% to compile your LaTeX source file. If your tool does not support these magic comments,
%% you will need to make appropriate manual choices.
%% 
%% You can safely use "pdflatex" instead of "xelatex" if you prefer the pdfLaTeX toolchain.
%% However, pdfLaTeX will not be able to deliver the professional font experience that you
%% will get with XeLaTeX.
%% 
%% _Important_: These magic comments should be on the first lines of your source file.
%% 
%%%%%%%%%%%%%%%%%%%%%%%%%%%%%%%%%%%%%%%%%%%%%%%%%%%%%%%%%%%%%%%%%%%%%%%%%%%%%%%%

%%%%%%%%%%%%%%%%%%%%%%%%%%%%%%%%%%%%%%%%%%%%%%%%%%%%%%%%%%%%%%%%%%%%%%%%%%%%%%%%
%% 
%%            JJJJ   K                         K   UUUU         UUUU  
%%            JJJJ   KKKK                   KKKK   UUUU         UUUU  
%%            JJJJ   KKKKKK               KKKKKK   UUUU         UUUU  
%%            JJJJ      KKKKKK         KKKKKK      UUUU         UUUU  
%%            JJJJ         KKKKKK   KKKKKK         UUUU         UUUU  
%%            JJJJ            KKKKKKKKK            UUUU         UUUU  
%%    JJ     JJJJJ               KKK               UUUUU       UUUUU  
%%  JJJJJJJJJJJJJ    KKKKKKKKKKKKKKKKKKKKKKKKKKK    UUUUUUUUUUUUUUU   
%%    JJJJJJJJJ      KKKKKKKKKKKKKKKKKKKKKKKKKKK      UUUUUUUUUUU     
%% 
%% This is an example file for using the JKU LaTeX technical report template for
%%  your technical report.
%% 
%% Template created by Michael Roland (2021)
%% 
%%%%%%%%%%%%%%%%%%%%%%%%%%%%%%%%%%%%%%%%%%%%%%%%%%%%%%%%%%%%%%%%%%%%%%%%%%%%%%%%

%%%%%%%%%%%%%%%%%%%%%%%%%%%%%%%%%%%%%%%%%%%%%%%%%%%%%%%%%%%%%%%%%%%%%%%%%%%%%%%%
%% 
%% Document class: This is a koma-script article.
%% 
\documentclass[a4paper,oneside,10pt,ngerman,english]{scrartcl}
%% 
%% The comma-separated list in square brackets are class options.
%% Useful options that you might want to use:
%% 
%% Paper size:
%%  * a4paper ... A4 paper size
%% 
%% Optimize for single-sided or double-sided printing:
%%  * oneside ... single-sided
%%  * twoside ... double-sided
%% 
%% Base font size:
%%  * 10pt ... 10-pt font is used for normal text
%%  * 11pt ... 11-pt font is used for normal text
%% 
%% Define document languages (the last specified language becomes the document default
%% language):
%%  * ngerman ... German
%%  * english ... English
%%  * ...
%% 
%% Alternate document classes: The JKU report template supports the koma-script classes
%% `scrartcl', `scrreprt' and `scrbook'. The article class `scrartcl' is well-suited
%% for a typical technical report. However, `scrbook' or `scrreprt' may be better
%% suited for longer reports since they permit structuring your work in chapters.
%%  
%% _Important_: The document class should be the first line of LaTeX code in your main
%% source file. Do not place anything but comments / magic comments above that line (unless
%% you really know what you are doing).
%% 
%%%%%%%%%%%%%%%%%%%%%%%%%%%%%%%%%%%%%%%%%%%%%%%%%%%%%%%%%%%%%%%%%%%%%%%%%%%%%%%%

%%%%%%%%%%%%%%%%%%%%%%%%%%%%%%%%%%%%%%%%%%%%%%%%%%%%%%%%%%%%%%%%%%%%%%%%%%%%%%%%
%% 
%% Treat input files as UTF-8 encoded. Make sure to always load this when you use pdfLaTeX
%% so that pdfLaTeX knows how to read and interpret characters this source file.
%% 
\usepackage[utf8]{inputenc}
%% 
%%%%%%%%%%%%%%%%%%%%%%%%%%%%%%%%%%%%%%%%%%%%%%%%%%%%%%%%%%%%%%%%%%%%%%%%%%%%%%%%

%%%%%%%%%%%%%%%%%%%%%%%%%%%%%%%%%%%%%%%%%%%%%%%%%%%%%%%%%%%%%%%%%%%%%%%%%%%%%%%%
%% 
%% Use the JKU LaTeX technical report template for this document.
%% 
\usepackage[techreport,fancyfonts]{jkureport}
%% 
%% The comma-separated list in square brackets are theme options. Useful options that you
%% might want to use:
%% 
%% Document type:
%%  * phdthesis     ... PhD thesis.
%%  * mathesis      ... Master's thesis.
%%  * diplomathesis ... Diploma thesis.
%%  * bathesis      ... Bachelor's thesis.
%%  * seminarreport ... Seminar report.
%%  * techreport    ... Technical report.
%% 
%% Color scheme selection options:
%%  * JKU  ... Use JKU (gray) color scheme (this is the default if no scheme is selected).
%%  * BUS  ... Use Business School color scheme.
%%  * LIT  ... Use Linz Institute of Technology color scheme.
%%  * MED  ... Use MED faculty color scheme.
%%  * RE   ... Use RE faculty color scheme.
%%  * SOE  ... Use School of Education color scheme.
%%  * SOWI ... Use SOWI faculty color scheme.
%%  * TNF  ... Use TNF faculty color scheme.
%% 
%% Space-efficient monospace font options (requires XeTeX):
%%  * compactmono   ... Use condensed fixed-width font everywhere.
%%  * nocompactverb ... Do not use condensed fixed-width font for verbatim and listings.
%% 
%% Style-breaking options:
%%  * noimprint      ... Do not insert imprint on title pages.
%%  * notitlelogo    ... Do not insert JKU logos on title pages.
%%  * capstitle      ... Set document title in capital letters.
%%  * nofancyfonts   ... Do not use custom TTF fonts with XeTeX / supress pdfLaTeX warning.
%%  * equalmargins   ... Decrease the outer page margin to have both page margins of equal size
%%                       (the additional outer margin is intentional and to be used for
%%                       anotations; equalmargins also causes the text width to be
%%                       significantly larger than optimal for reading).
%% 
%% Experimental options:
%%  * mathastext ... Use standard document fonts (and default to sans-serif font) in math mode.
%% 
%% Advanced options:
%%  * noautopdfinfo     ... Do not automatically try to add pdfinfo with hyperref from document
%%                          metadata fields.
%%  * logopath={<path>} ... Set the path where the theme can find its own logo resources. This
%%                          should typically be a relative path and the default is `./logos'.
%%  * fontpath={<path>} ... Set the path where the theme can find its own font resources. This
%%                          should typically be a relative path and the default is `./fonts'.
%% 
%% Hint: Boolean options can be used in the forms `option' or `option=true' the enable the
%% option and `nooption' or `option=false' to disable the option.
%% 
%%%%%%%%%%%%%%%%%%%%%%%%%%%%%%%%%%%%%%%%%%%%%%%%%%%%%%%%%%%%%%%%%%%%%%%%%%%%%%%%

%%%%%%%%%%%%%%%%%%%%%%%%%%%%%%%%%%%%%%%%%%%%%%%%%%%%%%%%%%%%%%%%%%%%%%%%%%%%%%%%
%% 
%% This is the place where you can load additional packages. If you want to load
%% a package `biblatex', you would use the command `\usepackage{biblatex}'.
%% 

\usepackage{csquotes}
\usepackage[backend=biber,citestyle=numeric,sortcites=true,maxcitenames=2,style=ACM-Reference-Format]{biblatex}
\setcounter{biburlnumpenalty}{100} %% reducing biburl* penalties typically improves URL placement in bibliography
\setcounter{biburllcpenalty}{100}
\setcounter{biburlucpenalty}{100}
\usepackage{todonotes}
\usepackage{import}
\usepackage{amsfonts}
\usepackage{subfigure}

%% 
%%%%%%%%%%%%%%%%%%%%%%%%%%%%%%%%%%%%%%%%%%%%%%%%%%%%%%%%%%%%%%%%%%%%%%%%%%%%%%%%

%%%%%%%%%%%%%%%%%%%%%%%%%%%%%%%%%%%%%%%%%%%%%%%%%%%%%%%%%%%%%%%%%%%%%%%%%%%%%%%%
%% 
%% Bibliography data files.
%% 

\addbibresource{references.bib}

%% 
%%%%%%%%%%%%%%%%%%%%%%%%%%%%%%%%%%%%%%%%%%%%%%%%%%%%%%%%%%%%%%%%%%%%%%%%%%%%%%%%

\begin{document}
%%%%%%%%%%%%%%%%%%%%%%%%%%%%%%%%%%%%%%%%%%%%%%%%%%%%%%%%%%%%%%%%%%%%%%%%%%%%%%%%
%% 
%% Report information and title page
%% 

%% Command \title{title}: sets the title of your report
\title{Leveraging the P Extension for accelerating Matrix Multiplication}

%% Command \titleshort{short title}: sets an abbreviated version of the report title for page heads
%\titleshort{Optional space for your abbreviated title}

%% Command \author{name}: sets the author's name; use \prefix{} and \suffix{} to add academic titles and suffixes, use \matno{} to add the immatriculation number
\author{%
	Jonas Karg
    \affiliation{Institute for Complex~Systems}
    \authornewline
    \authormail{k12213152@jku.at}
    \authorweb{https://jku.at/ics}
    \authornewline
}

% Command \date{YYYY-MM-DD}: set the day of publication (defaults to today)
%\date{2020-04-09}

% Command \partnerlogo{filename}: use filename as partnerlogo, filename may be blank to disable the logo
%\partnerlogo{logos/ins}

% Command \revisionblock{text}: set the document revision block on the title page
\revisionblock{Space for your revision block, acknowledgements, etc.}

% Command \reportnumber{number}: set the report number
%\reportnumber{Space for your report number}

% Command \setbottommark{text}: set the bottom mark (in document footer)
%\setbottommark{Space for your bottom mark}

% Command \abstract{text}: set the document abstract on the title page
\abstract{Space for your (short) abstract.}

% Command \keywords{text}: set the document keywords
%\keywords{Space for your comma-separated keywords}


%% Finally, print the title page using the above information:
\maketitle
%% 
%%%%%%%%%%%%%%%%%%%%%%%%%%%%%%%%%%%%%%%%%%%%%%%%%%%%%%%%%%%%%%%%%%%%%%%%%%%%%%%%

%%%%%%%%%%%%%%%%%%%%%%%%%%%%%%%%%%%%%%%%%%%%%%%%%%%%%%%%%%%%%%%%%%%%%%%%%%%%%%%%
%% 
%% Add a table of contents
%% 

%% Make sure to start the table of contents on a new odd page (odd is only relevant in twoside layout)
\cleardoubleoddpage
%% Print the table of contents
\tableofcontents

%% 
%%%%%%%%%%%%%%%%%%%%%%%%%%%%%%%%%%%%%%%%%%%%%%%%%%%%%%%%%%%%%%%%%%%%%%%%%%%%%%%%

%%%%%%%%%%%%%%%%%%%%%%%%%%%%%%%%%%%%%%%%%%%%%%%%%%%%%%%%%%%%%%%%%%%%%%%%%%%%%%%%
%% 
%% Abstract: Instead of an abstract on the title page (see \abstract{...}), you
%% sometimes want to add an abstract as its own unnumbered section.
%% 

%% (Optionally) let the abstract start on a new odd page (odd is only relevant in twoside layout)
\cleardoubleoddpage

\addsec{Abstract}

This document is intended to accompany a project where open-source software was used to compile and simulate a handful of matrix multiplication algorithms that employ functionality provided by the RISC-V P-extension for acceleration. The following pages are used to give a brief overview over the P-extension, some ways to accelerate the naive matrix multiplication algorithm, my process for compiling and simulating the implementation, as well as for reviewing the results.


%% 
%%%%%%%%%%%%%%%%%%%%%%%%%%%%%%%%%%%%%%%%%%%%%%%%%%%%%%%%%%%%%%%%%%%%%%%%%%%%%%%%

%%%%%%%%%%%%%%%%%%%%%%%%%%%%%%%%%%%%%%%%%%%%%%%%%%%%%%%%%%%%%%%%%%%%%%%%%%%%%%%%
%% 
%% Add your report sections ...
%% 

%% (Optionally) let the main sections start on a new odd page (odd is only relevant in twoside layout)
\cleardoubleoddpage

\section{Introduction}
\label{sec:introduction}

The goal of this paper was to implement common algorithms used in machine learning using the functionality of the RISC-V P-extension to parallelize certain aspects, and later compare the performance of these algorithms to non-SIMD implementations using different metrics such as throughput and number of instructions.

The motivation behind this paper was my personal fascination with the field of machine learning. Machine learning workloads have, over the past few years, led to a whole new category of interesting computational problems that more often than not profit massively from new kinds of hardware.

RISC-V is particularly compelling for machine learning applications because its extensibility enables tailored solutions for the rapidly evolving computational demands of the field. Machine learning workloads often benefit from specialized hardware to optimize performance for tasks such as matrix multiplication, convolution or reduction operations. RISC-V's open architecture allows developers to create custom extensions and accelerators, allowing adaptability to new algorithms and techniques.

I consider the P-extension a low-hanging fruit for toying around in this field, for one because this extension is now at a point where, even though it is not yet ratified, we have some support for compilers through a downstream version of the \emph{riscv-gnu-toolchain}, as well as environments where P-extension functionality can be simulated e.g. \emph{riscv-isa-sim} (spike). Another reason why packed SIMD instructions are interesting for machine learning in particular, is a recent trend in the development of machine learning models, where models internally use lower numerical precision. In other words, it has become apparent that in many places, lowering numerical precision oftentimes has little to no effect on the performance of a model, which means we can get away with using quantization levels going from the more common 16- or 32-bits, down to only four bits per parameter (although ultra-low quantization usually requires more significant changes to model architecture). This way, we could parallelize computation using packed SIMD in certain areas during inference and/or training, which means we might be able to gain performance while lowering power requirements in the process.


\section{An overview over the P-extension}
\label{sec:overview}
The "P" extension introduces support for \emph{packed} SIMD instructions to the RISC-V instruction set. SIMD stands for “Single Instruction Multiple Data” and allows, as the name suggests, parallel execution of a single instruction over multiple separate data elements. This is sometimes called vector processing, and such capabilities are commonly used to accelerate digital signal processing (DSP), and multimedia applications, as well as a range of other tasks. A classic example of a use case where one can gain immediate benefits from using SIMD is pairwise operations, such as addition of two arrays of numbers. For two arrays array of length n with a SIMD instruction set that allows parallel addition of $ m $ elements from array $ A $ and $ m $ elements from array $ B $ at once, we would theoretically only need $ \frac{n}{m} $ instructions to compute the result instead of $ n $ instructions without SIMD, which would speed up the addition by a factor of $ m $. Of course, we can only expect to get similar performance gains in a very narrow set of scenarios, and even then there are factors like memory access patterns, that can diminish performance gains. However, when used carefully, SIMD can still provide significant real-world performance gains
\cite{simd_access_patterns}.

For x86 processors, there are the “Advanced Vector Extensions” (AVX/AVX2) developed by Intel, which are supported by almost all modern intel and AMD CPUs. Aside from AVX, there is a plethora of other SIMD extensions for a wide variety of platforms. Most of the modern SIMD extensions use custom registers for SIMD operations, which are usually larger than the general purpose registers, to be able to fit multiple full-precision values into them. AVX for example supports up to 512 bit wide SIMD registers \cite{computerbase_avx_2008, intel_avx512_2023}.

\begin{table}
\centering
\caption{
    A table. Be aware that tables have their caption above the table
}\label{tab:nicetable}
\begin{tabularx}{\linewidth}{l>{\raggedright\arraybackslash}X}
\toprule
\textbf{Option}        & \textbf{Description} \\
\midrule
\texttt{phdthesis}     & PhD thesis \\
\texttt{mathesis}      & Master's thesis \\
\texttt{diplomathesis} & Diploma thesis \\
\texttt{bathesis}      & Bachelor's thesis \\
\texttt{seminarreport} & Seminar report \\
\texttt{techreport}    & Technical report \\
\bottomrule
\end{tabularx}
\end{table}


\begin{itemize}
\item The Tor directory authority moria1 shows a voting behavior for the HSDir flag that significantly deviates from that of other directory authorities~\cite{bib:2021-hoeller-iiwas}. Be aware that the \string\cite\ command is part of the sentence and, hence, comes before the period.
\item \citeauthor{bib:2015-roland-thesis-book}~\cite{bib:2015-roland-thesis-book} proposes a novel attack concept against NFC secure elements in mobile phones.
\item \citeauthor{bib:2013-roland-woot}~\cite{bib:2013-roland-woot} uncovered a flaw in legacy-support of the MasterCard contactless payment protocols that allows an attacker to clone certain credit cards.
\item \citeauthor{bib:2021-hoeller-foci}~\cite{bib:2021-hoeller-foci} designed an experiment to measure the usage of Tor V3 onion services in a privacy-conscious way. Here we used ``et al.'' because the cited work has more than two authors (\string\citeauthor\ will automatically take care of this).
\item \citeauthor{bib:2021-hoeller-iiwas}~\cite{bib:2021-hoeller-iiwas} conclude:
\begin{quote}
	
\emph{Ultimately, the high fluctuations in the hidden service directory were caused by a mixture of several issues. First the changed voting behavior of three directory authorities reduced the amount of obtainable votes to six. If any of the remaining six relays went offline -- which tends to happen during ongoing DOS attacks -- relays needed to obtain five out of five available votes. So any individual measurement failure regarding either bandwidth or uptime led to a withdrawn HSDir flag.}

\end{quote}
\item You sometimes also paraphrase from multiple sources. For that purpose, the \string\cite\ command accepts a list of multiple comma-separated references. Do not add spaces inbetween them. Various analyses of the Tor network have been performed recently~\cite{bib:2021-hoeller-foci,bib:2021-hoeller-iiwas}. 
\end{itemize}



\section{Conclusion}
\label{sec:conclusion}

Space for your summary, central conclusions, and an outlook on potential future work.


%% 
%%%%%%%%%%%%%%%%%%%%%%%%%%%%%%%%%%%%%%%%%%%%%%%%%%%%%%%%%%%%%%%%%%%%%%%%%%%%%%%%

%%%%%%%%%%%%%%%%%%%%%%%%%%%%%%%%%%%%%%%%%%%%%%%%%%%%%%%%%%%%%%%%%%%%%%%%%%%%%%%%
%% 
%% Print the bibliography
%% 
%% Make sure to start the bibliography on a new odd page (odd is only relevant in twoside layout)
%\cleardoubleoddpage
\printbibliography
%% 
%%%%%%%%%%%%%%%%%%%%%%%%%%%%%%%%%%%%%%%%%%%%%%%%%%%%%%%%%%%%%%%%%%%%%%%%%%%%%%%%

%% Begin with the appendix part (all further sections will be appendices)
\appendix

%%%%%%%%%%%%%%%%%%%%%%%%%%%%%%%%%%%%%%%%%%%%%%%%%%%%%%%%%%%%%%%%%%%%%%%%%%%%%%%%
%% 
%% Add your appendix sections ...
%% 

%% Make sure to start the appendix on a new odd page (odd is only relevant in twoside layout)
%\cleardoubleoddpage
\section{An Appendix}
\label{app:an-appendix}

Space for an appendix.
You can have more than one appendix section.
Appendices are, of course, optional.


%% 
%%%%%%%%%%%%%%%%%%%%%%%%%%%%%%%%%%%%%%%%%%%%%%%%%%%%%%%%%%%%%%%%%%%%%%%%%%%%%%%%
\cleardoubleoddpage

\end{document}
\endinput
